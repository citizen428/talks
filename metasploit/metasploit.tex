% Created 2012-04-25 Wed 16:51
\documentclass[11pt]{beamer}

               \mode<presentation>

               \usetheme{Berlin}

               \usecolortheme{beetle}

               \beamertemplateballitem

               \setbeameroption{show notes}
               \usepackage[utf8]{inputenc}

               \usepackage[T1]{fontenc}

               \usepackage{hyperref}

               \usepackage{color}
               \usepackage{listings}
               \lstset{numbers=none,language=[ISO]C++,tabsize=4,
             frame=single,
             basicstyle=\small,
             showspaces=false,showstringspaces=false,
             showtabs=false,
             keywordstyle=\color{blue}\bfseries,
             commentstyle=\color{red},
             }

               \usepackage{verbatim}

               \institute{Linuxwochenende 2011}

                \subject{{{{beamersubject}}}}

\usepackage[utf8]{inputenc}
\usepackage[T1]{fontenc}
\usepackage{fixltx2e}
\usepackage{graphicx}
\usepackage{longtable}
\usepackage{float}
\usepackage{wrapfig}
\usepackage{soul}
\usepackage{textcomp}
\usepackage{marvosym}
\usepackage{wasysym}
\usepackage{latexsym}
\usepackage{amssymb}
\usepackage{hyperref}
\tolerance=1000
\providecommand{\alert}[1]{\textbf{#1}}

\title{Penetrationstests mit Metasploit}
\author{Michael Kohl}
\date{\today}
\hypersetup{
  pdfkeywords={},
  pdfsubject={},
  pdfcreator={Emacs Org-mode version 7.8.03}}

\begin{document}

\maketitle

\begin{frame}
\frametitle{Outline}
\setcounter{tocdepth}{3}
\tableofcontents
\end{frame}

\section{Einleitung}
\label{sec-1}
\begin{frame}[fragile]\frametitle{Über mich}
\label{sec-1-1}
\begin{itemize}

\item Früher: Linux/Unix Admin / Systems Engineer\\
\label{sec-1-1-1}%
\item Jetzt: Rails-Entwickler, DevOps\\
\label{sec-1-1-2}%
\item Ab Oktober: Penetration Tester\\
\label{sec-1-1-3}%
\item Gentoo Entwickler, Metalab, RubyLearning, etc.\\
\label{sec-1-1-4}%
\end{itemize} % ends low level
\end{frame}
\begin{frame}[fragile]\frametitle{Warum Penetration Testing?}
\label{sec-1-2}
\begin{itemize}

\item Schwachstellen identifizieren\\
\label{sec-1-2-1}%
\item Aufdecken von Fehlern aus falscher Bedienung\\
\label{sec-1-2-2}%
\item Erhöhung der Sicherheit auf technischer und organisatorischer Ebene\\
\label{sec-1-2-3}%
\item externe Validierung der Sicherheit\\
\label{sec-1-2-4}%
\end{itemize} % ends low level
\end{frame}
\section{Penetration Testing}
\label{sec-2}
\begin{frame}[fragile]\frametitle{Pre-Engagement}
\label{sec-2-1}
\begin{itemize}

\item Zieldefinition (z.B. compliance)\\
\label{sec-2-1-1}%
\item Abklären der Rahmenbedingungen (z.B. overt/covert)\\
\label{sec-2-1-2}%
\item Limitierungen (z.B. nur Kernarbeitszeit, Wochenenden)\\
\label{sec-2-1-3}%
\item Umfang (Netzwerk, Apps, WLAN, physische Tests, Social Engineering)\\
\label{sec-2-1-4}%
\item Kommunikationswege definieren\\
\label{sec-2-1-5}%
\end{itemize} % ends low level
\end{frame}
\begin{frame}[fragile]\frametitle{Intelligence Gathering}
\label{sec-2-2}
\begin{itemize}

\item Versuch möglichst viel über Ziel herauszufinden\\
\label{sec-2-2-1}%
\item Social Media\\
\label{sec-2-2-2}%
\item Footprinting\\
\label{sec-2-2-3}%
\item Portscans\\
\label{sec-2-2-4}%
\item Firewalls etc?\\
\label{sec-2-2-5}%
\item physische Locations\\
\label{sec-2-2-6}%
\end{itemize} % ends low level
\end{frame}
\begin{frame}[fragile]\frametitle{Threat Modeling}
\label{sec-2-3}
\begin{itemize}

\item benutzt Informationen aus dem vorherigen Schritt\\
\label{sec-2-3-1}%
\item Versuch vielversprechendsten Angriffsvektor zu finden\\
\label{sec-2-3-2}%
\item in die Rolle des Angreifers versetzen
\label{sec-2-3-3}%
\begin{itemize}

\item Analyse von Assets\\
\label{sec-2-3-3-1}%
\item Analyse der Geschäftsprozesse\\
\label{sec-2-3-3-2}%
\item Unternehmenstruktur\\
\label{sec-2-3-3-3}%
\item Attacken auf ähnliche Unternehmen\\
\label{sec-2-3-3-4}%
\end{itemize} % ends low level
\end{itemize} % ends low level
\end{frame}
\begin{frame}[fragile]\frametitle{Vulnerability Analysis}
\label{sec-2-4}
\begin{itemize}

\item Port- und Service-Scans\\
\label{sec-2-4-1}%
\item Banner Grabbing\\
\label{sec-2-4-2}%
\item SQL Injection Scanner\\
\label{sec-2-4-3}%
\item Traffic Monitoring\\
\label{sec-2-4-4}%
\end{itemize} % ends low level
\end{frame}
\begin{frame}[fragile]\frametitle{Exploitation}
\label{sec-2-5}
\begin{itemize}

\item ``spektakulärste'' Phase\\
\label{sec-2-5-1}%
\item nach Indentifikation der vielversprechenden Vektoren\\
\label{sec-2-5-2}%
\item Exploits für bekannte Versionen\\
\label{sec-2-5-3}%
\item Buffer Overflows, SQL Injections, Passwort Bruteforce etc.\\
\label{sec-2-5-4}%
\end{itemize} % ends low level
\end{frame}
\begin{frame}[fragile]\frametitle{Post Exploitation}
\label{sec-2-6}
\begin{itemize}

\item nach dem Kompromittieren eines oder mehrerer Systeme\\
\label{sec-2-6-1}%
\item Identifikation wichtiger Infrastruktur\\
\label{sec-2-6-2}%
\item Identifikation wichtigster Daten\\
\label{sec-2-6-3}%
\item Schwachstellen mit grösstem Business Impact\\
\label{sec-2-6-4}%
\item Aufräumen\\
\label{sec-2-6-5}%
\end{itemize} % ends low level
\end{frame}
\begin{frame}[fragile]\frametitle{Reporting}
\label{sec-2-7}
\begin{itemize}

\item wichtigster Teil\\
\label{sec-2-7-1}%
\item was?\\
\label{sec-2-7-2}%
\item wie?\\
\label{sec-2-7-3}%
\item wie reparieren?\\
\label{sec-2-7-4}%
\item generelle Security, nicht nur technische Schwachstellen\\
\label{sec-2-7-5}%
\end{itemize} % ends low level
\end{frame}
\section{Metasploit}
\label{sec-3}
\begin{frame}[fragile]\frametitle{Metasploit Framework}
\label{sec-3-1}
\begin{itemize}

\item Penetration Testing Framework in Ruby\\
\label{sec-3-1-1}%
\item Scanner/Fuzzer\\
\label{sec-3-1-2}%
\item Payloads\\
\label{sec-3-1-3}%
\item Exploits\\
\label{sec-3-1-4}%
\item Post-Exploitation Tools (Meterpreter)\\
\label{sec-3-1-5}%
\item Libraries zur Entwicklung eigener Tools\\
\label{sec-3-1-6}%
\end{itemize} % ends low level
\end{frame}
\section{Demo}
\label{sec-4}
\begin{frame}[fragile]\frametitle{Demo}
\label{sec-4-1}
\begin{itemize}

\item Metasploitable VM\\
\label{sec-4-1-1}%
\item prinzipieller Ablauf, kein vollständiger Pen Test\\
\label{sec-4-1-2}%
\end{itemize} % ends low level
\end{frame}
\section{Ressourcen}
\label{sec-5}
\begin{frame}[fragile]\frametitle{Ressourcen}
\label{sec-5-1}
\begin{itemize}

\item \href{http://de.wikipedia.org/wiki/Penetrationstest_(Informatik)}{http://de.wikipedia.org/wiki/Penetrationstest\_(Informatik)}\\
\label{sec-5-1-1}%
\item \href{http://www.pentest-standard.org/index.php/Main_Page}{http://www.pentest-standard.org/index.php/Main\_Page}\\
\label{sec-5-1-2}%
\item \href{http://www.metasploit.com/}{http://www.metasploit.com/}\\
\label{sec-5-1-3}%
\item \href{http://www.offensive-security.com/metasploit-unleashed/}{http://www.offensive-security.com/metasploit-unleashed/}\\
\label{sec-5-1-4}%
\item \href{http://nostarch.com/metasploit}{http://nostarch.com/metasploit}\\
\label{sec-5-1-5}%
\item \href{http://www.offensive-security.com/metasploit-unleashed/Metasploitable}{http://www.offensive-security.com/metasploit-unleashed/Metasploitable}\\
\label{sec-5-1-6}%
\end{itemize} % ends low level
\end{frame}

\end{document}
